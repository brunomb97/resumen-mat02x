\documentclass[letterpaper,12pt]{book}
\usepackage[utf8]{inputenc}

\usepackage{geometry}
\geometry{letterpaper}

\usepackage{parskip}

\usepackage[brazil,spanish]{babel}           
\usepackage[utf8]{inputenc}

\usepackage{changepage}
\usepackage[usenames]{color}
\usepackage{graphicx}
\usepackage{graphics}
\usepackage{subfiles}
\usepackage{array}
\usepackage{enumitem}
\usepackage{multirow}
\usepackage{float}
\usepackage{calc}
\usepackage{hyperref}



\usepackage{bbm}
\usepackage[notquote]{hanging}
\usepackage{amsmath}
\usepackage{mathrsfs}
\usepackage{amsfonts}
\usepackage{mathtools}
\usepackage{amsthm}
\usepackage{amsmath}
\usepackage{amssymb}
\usepackage{amstext}
\usepackage{latexsym}
\usepackage{dsfont}
\usepackage[framemethod=TikZ]{mdframed}

\usepackage{fancyhdr}

\pagestyle{fancy}
\lhead{}
\chead{}
\rhead{\thepage}
\renewcommand{\headrulewidth}{0pt}
\lfoot{}
\cfoot{}
\rfoot{}

% Comandos personalizados
\usepackage{custom}

\title{\Titulo - \Sigla \Ramo}
\author{\Name}

\begin{document}
\thispagestyle{empty}

\begin{center}
    \huge{\textsc{Resumen MAT02X}}
\end{center}

\tableofcontents

\chapter{MAT021}
    \section{Trigonometría}
        \subfile{mat021/trigonometria/trigonometria.tex}
    \section{Sumatorias y Series}
        \subfile{mat021/series/series.tex}
    \section{Límites}
        \subfile{mat021/limites/limites.tex}
    \section{Derivadas}
        \subfile{mat021/derivadas/derivadas.tex}
    \section{Lugares Geométricos}
        \subfile{mat021/lugares-geometricos/lugares-geometricos.tex}
    \section{Lógica}
        \subfile{mat021/logica/logica.tex}
    \section{Matrices}
        \subfile{mat021/matrices/matrices.tex}
    \section{Números Complejos}
        \subfile{mat021/complejos/complejos.tex}
    \section{Fracciones Parciales}
        \subfile{mat021/fracciones-parciales/fracciones-parciales.tex}
    \section{Funciones Paramétricas}
        \subfile{mat021/funciones-parametricas/funciones-parametricas.tex}
    \section{Funciones Polares}
        \subfile{mat021/funciones-polares/funciones-polares.tex}
        
\chapter{MAT022}
    \section{Antiderivadas e Integrales}
\vspace{1em}

\end{document}
